Much of everyday language is vague, even in situations where vagueness could have been avoided (i.e., where vagueness is used ``strategically''). Yet the benefits of vagueness for hearers and readers are proving to be elusive. We discuss a range of earlier controlled experiments with human participants, and we report on a new series of experiments that we conducted in recent years. These experiments, which focus on vague expressions that are part of referential noun phrases, aim to separate the utility of vagueness (as defined by the existence of borderline cases) from the utility of other factors that tend to co-occur with vagueness. Having presented the evidence, we argue that the evidence supports a view where the benefits that vague terms exert are due to other influences, and not to vagueness itself.
