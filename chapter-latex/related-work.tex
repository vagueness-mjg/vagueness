A number of answers to Lipman's question have been proposed, but as we have argued elsewhere, it appears that some of these do not to hold up to scrutiny \citep{van2009utility,vanDeemterBook}. For example,
%
\begin{itemize}
\item It has been proposed that the existence of vague words makes a language more efficient \citep{barwiseperry}. The idea is that vagueness allows the use of one and the same word (e.g., ``big'') in different situations, which require different standards. For instance, the vague word ``big'' can denote a mouse a well as an elephant. The idea is plausible, but the problem is that ``big" is efficient not because it is vague (i.e., not because it allows borderline cases) but because it is context dependent. To see this, note for example that superlatives are not vague, but they {\em are} context dependent: ``the biggest $x$" ascribes different sizes to the referent depending on $x$. It is for this reason that we can use the word ``biggest''  to talk about the biggest mouse, the biggest elephant, and so on. Context-dependence, however, does not imply vagueness.
%
\item It has been pointed out that vague words are capable of combining a statement of quantity (along some dimension) with an evaluative statement \citep{veltman}. The idea is that when we say ``The patient has a fever'', we do not merely assert that her temperature lies above a certain value, we also imply that the deviation is large enough to be clinically significant (i.e., something is not right with the patient). The problem with this plausible idea is that evaluation does not imply vagueness. For example, the medical term ``obese'' is evaluative (i.e., it is not healthy to be obese), yet its standard medical definition in terms of Body Mass Index is perfectly crisp, with BMI values above 30 counting as obese.
\end{itemize}
%
Analogous observations can be made about a number of other purported benefits of vagueness; one of the few hypotheses left standing at the moment is the {\em cost reduction} hypothesis, which we mentioned in the previous section. The findings on which we are reporting in the present paper will follow a familiar pattern: we will explore the cost reduction hypothesis, only to conclude that it does not stand up to scrutiny. Our approach, this time, will be experimental: we conduct controlled experiments to investigate the effect that vague expressions have on a hearer. 

Charting the utility of vagueness is also the attested aim of a small number of previous experimental studies. But, as it happens, few of these studies have truly focussed on vagueness in the sense on which we are focussing (i.e., they did not address Lipman's challenge). Two recent studies illustrate this issue; let's discuss them briefly. 

In a series of studies of behaviour modification, \citet{Mishra01042011} manipulated the presentation format of information about quantities in the domains of mental acuity, physical strength, and weight loss. In the weight loss study, participants were told that the study was designed to test the validity of a new (actually fictitious) health index, the HHI (Holistic Health Index). They were told that an ideal HHI score lies in the range of 45 to 55. In a longitudinal study, participants submitted their weight to a computer each week. Participants were told that two algorithms would be used to compute their HHI, and that the two might give different values initially, in which case the true score lay between the two values. In one condition, which the authors called the precise condition, the two algorithms gave the same score. In the other condition, which the authors called the vague condition, one algorithm added 3\% to the score while the other algorithm subtracted 3\% from the score, yielding a range of values whose midpoint was the same as the two values given in the precise condition. 

One group of participants was given HHI scores in the ideal range: for this group their weight loss did not differ depending on whether they were given vague or precise HHI values. However for the other group, who were given HHI scores outside the ideal range, their weight loss was significantly greater if they were given vague HHI scores than if they were given precise HHI scores. The authors explain the improvement in the vague condition for this group as resulting from the participants' freedom to think of themselves a positioned on one end of the range - the end closest to the ideal HHI scores. This ``illusion of proximity'' \citep[][p.~4]{Mishra01042011} to the goal is argued to allow participants to generate positive expectancies that lead to behaviours that improve performance. In contrast, in the precise conditions, participants did not have this freedom of interpretation, and could not distort the information to bring about the beneficial \emph{illusion of proximity}. These results are interesting, and of obvious potential practical importance. We note, however, that information presented as an exact range of values does not conform with the standard definition of vagueness \citep{keefe1997vagueness, EgreKlinedinst}, since an exact range does not admit borderline cases. In the terminology of \citeauthor{Hobbs85granularity}, the difference between a range and a single midpoint value is a difference of \emph{granularity}. Furthermore, the experiments of \citeauthor{Mishra01042011} did not explore benefits in terms of processing cost, but in terms of long-term behaviour change.

Similar issues arise from the work of \citet{peters2009bringing}. The authors carried out a series of studies where participants were required to rate hospitals based on various sources of information about quality of care. There was a between-subjects manipulation based on numeracy. The format of the information was manipulated within subjects: either numbers only were presented, or both numbers and evaluative categories were presented (e.g., \emph{Poor}, \emph{Fair}, \emph{Good}, \emph{Excellent}, with crisp visual boundary lines between the categories). Results showed that, for low-numeracy participants, the presence of evaluative categories resulted in a diminished influence of an irrelevant affective state on the ratings. For all participants, the presence of evaluative categories resulted in better decisions and in a greater use of the most important and reliable types of information, such as survival rates. 

It is, however, questionable whether the ``evaluative categories'' manipulation in this study can be considered a manipulation of vagueness. Certainly, terms like \emph{Fair} admit the possibility of borderline cases. However, given that the boundaries between the categories were marked crisply, and that therefore the categories mapped crisply to numerical values, it becomes doubtful whether any borderline cases could be conceived to arise in fact. For example, \emph{Fair} was mapped to 60\% -- 70\% for the variable \emph{percentage of heart attack patients given recommended treatment (ACE inhibitor)}. Accordingly, rather than the vagueness of categories such as \emph{Poor}, Peters et al. emphasise the evaluative content inherent in these categories, and the affective potential of the evaluative content rather than the vagueness of the terms like {\em Fair}.
