\documentclass{beamer}
\usepackage[latin1]{inputenc}
\usepackage[T1]{fontenc}
\usepackage{beamerthemesplit}
\usepackage{graphics,epsfig, subfigure}
\usepackage{apacite}
\usepackage{amsfonts}
\newcommand{\tickYes}{\checkmark}
\usepackage{pifont}
\newcommand{\tickNo}{\hspace{1pt}\ding{55}}
%\usepackage{apalike}
\usepackage{url}

\definecolor{kugreen}{RGB}{50,93,61}
\definecolor{kugreenlys}{RGB}{132,158,139}
\definecolor{kugreenlyslys}{RGB}{173,190,177}
\definecolor{kugreenlyslyslys}{RGB}{214,223,216}
\setbeamercovered{transparent}
\mode<presentation>
{  \usetheme{PaloAlto}
  \usecolortheme[named=kugreen]{structure}
  \useinnertheme{circles}
  \usefonttheme[onlymath]{serif}
  \setbeamercovered{transparent}
  \setbeamertemplate{blocks}[rounded][shadow=true]
}
%\setbeamertemplate{background}{\includegraphics[width=1\textwidth]{natfak_baggrund.pdf}}
\logo{\includegraphics[width=1cm]{../logo/Crest_Colour}}

%\title{Et bachelorprojekt om roterende vand}
%\subtitle{Et heldigt tilf�lde}
%\author{Thomas R. N. Jansson}
%\institute{Niels Bohr Institute \\ University of Copenhagen}
%\date{9 November 2007}



% \usepackage{beamerthemesplit} // Activate for custom appearance

\title{The utility of vagueness}
\subtitle{an empirical test of the cost reduction hypothesis in a reference task}
\author{Matt Green \& Kees van Deemter}
\institute{NLG group \\ University of Aberdeen}
\date{7th June 2011}

\begin{document}

\frame{\titlepage}

\section[]{}
\frame{\frametitle{Outline}
\begin{enumerate}
\visible<1->{\item Rationale (what, why, how?)}
\visible<2->{\item Experiment one (tests the water)}
\visible<3->{\item Experiment two (extends the findings)}
\visible<4->{\item Interim summary (where do we stand now?)}
\visible<5->{\item Experiment three (addresses problems with E1 \& E2)}
\visible<6->{\item Interpretation (what does it all mean?)}
%\item Take home message
\end{enumerate}
%\tableofcontents}
}
\section[Rationale]{Rationale (what, why, how?)}
\frame{
            \frametitle{A definition of vagueness}
            Vagueness can be defined as the existence of borderline cases 
            \begin{itemize}
            \item e.g., \emph{tall}
            \end{itemize}}
 
%\section[Backgound]{Why study the effects of vagueness?}
\frame{
            \frametitle{Why study the effects of vagueness?}
            2 main reasons
            \begin{enumerate}
            \item Game theory needs an update
            \begin{itemize}
            \item Game-theory predicts that precision always `wins'
            \item But we use vague language a lot
            \item There must be some utility of vagueness that the models don't account for
            \end{itemize} 
           \visible<2->{ \item NLG systems need help
            \begin{itemize}
            \item Practical NLG systems must decide between generating vague and precise terms \cite<e.g., >[]{portet2009automatic}
            \item These decisions are (almost) guesswork currently
            \item Data from human comprehenders would provide a better basis for these decisions
            \item There isn't much data like this at the moment
            \end{itemize} }
            \end{enumerate}} 
            
%\section[]{Example NLG decision}
\frame{
            \frametitle{Example NLG decision}
            Which of these is better?
            \begin{enumerate}
%            \item ``By 10:29 there had been 2 successive spikes in TcPO2 up to 18.1"
%            \item ``There were a couple of spikes in the TcPO2 trace"
\item `Winds southwest $15$ to $25$ knots \textbf{becoming southeast 15} later this evening'
\item `Winds southwest $15$ to $25$ knots \textbf{diminishing to light} later this evening'
            \end{enumerate} 
            \visible<2->{\begin{itemize}
            \item The answer might depend on the audience: offshore oil rig, mountain resuce, flying doctor, air traffic control \ldots
            \end{itemize}}
            \visible<3>{\begin{itemize}
            \item It depends quite a lot what `better' means: speed, accuracy, recall, comprehension, effect on affect \ldots
            \end{itemize} } }

%\section{Vagueness in reference}
\frame{\frametitle{Vagueness in reference}
\begin{itemize}
	\item Vagueness in reference is highly context-dependent
	\item \underline{Potential} for vagueness is not necessarily \underline{realised}
\end{itemize} }

%\frame{\frametitle{Pinning vagueness down}
%Vagueness, or something else?\\
%\begin{itemize}
%\item Several, separate research comunities
%\begin{itemize} 
%\item Vagueness $\approx$ low specificity
%\item Vagueness $\approx$ large grain size
%\end{itemize}
%\end{itemize}
%}
%




\frame{\frametitle{Pinning utility down}
\begin{itemize}
\item Utility in \underline{co-operative} situations is our focus
\item Non-co-operative situations where vagueness has utility
	\begin{itemize}
	\item Sales, marketing, advertising \tickNo
	\item Politics, campaigning \tickNo
	\item Persuasion, behaviour modification \tickNo
	\end{itemize}
\item Co-operative situations where vagueness may have utility
	\begin{itemize}
	\item Local vagueness \tickNo
	\item Information reduction \tickNo
	\item Reducing cognitive load on the comprehender \tickYes
	\end{itemize}
\end{itemize}
}

\frame{\frametitle{The cost reduction hypothesis} 
``For the listener, information which is too specific may require more effort to analyze" \cite{lipmanvague}
}



\section[Test the water]{Experiment one (tests the water)}

\frame{\frametitle{Experiment one}
%            \frametitle{Stimulus}
            \begin{figure}
            \includegraphics{/Users/mjgmac562/Documents/vagueness-folder/writing/figures/drawingexp1b}
            \end{figure}}


\frame{\frametitle{Experiment one}
\begin{itemize}
\item Pilot experiment
\item $H_{0}$: \emph{No difference between vague and precise}
\item $H_{1}$: \emph{Some difference between vague and precise\\  ~~~~~ \ldots V better than P?}
\item Method: forced choice
\item DVs: RT, errors
\item IVs: Vagueness (Vague, Precise)
\item Number combinations: \{$2,4$\}, \{$2,6$\}, \{$3,5$\}, \{$5,9$\}, \{$6,8$\}, \{$7,3$\}, \{$7,9$\}, \{$8,4$\}
\end{itemize}
}



            
\frame{
	\frametitle{Results, experiment one}
	\begin{figure}
            \includegraphics[width=\textwidth]{/Users/mjgmac562/Documents/vagueness-folder/writing/figures/e1resultscolourpresentation}
            \end{figure}
	}


\section[Extend findings]{Experiment two (extends findings)}

\frame{\frametitle{Experiment two}
\begin{itemize}
\item Try larger numbers (not subitizable)
\item $H_{0}$: \emph{No difference between vague and precise}
\item $H_{1}$: \emph{Vagueness helps when the numbers are bigger, and when the difference between the numbers is greater}
\item Number combinations: \{$5,25$\} \{$10,25$\} \{$15,25$\} \{$20,25$\} \{$30,25$\} \{$35,25$\} \{$40,25$\}
\end{itemize}
}


\frame{\frametitle{Instruction first}
\begin{figure}
 \includegraphics[width=.8\textwidth]{/Users/mjgmac562/Documents/vagueness-folder/writing/figures/drawingexp2instruction}
\end{figure}
}

\frame{\frametitle{Stimulus next}
\begin{figure}
 \includegraphics[width=.8\textwidth]{/Users/mjgmac562/Documents/vagueness-folder/writing/figures/drawingexp2stimulusonly}
\end{figure}
}


\frame{
            \frametitle{Results, experiment two}
            \begin{figure}
            \includegraphics[width=\textwidth]{/Users/mjgmac562/Documents/vagueness-folder/writing/figures/e2resultscolourpresentation}
            \end{figure}
            }
     
\section[Where do we stand?]{Interim summary (where do we stand?)}
\frame{\frametitle{Interim summary (where do we stand?)}
\begin{itemize}
\item Vagueness is worse for very small numbers (expt 1)
\visible<2->{\item Vagueness is better for larger numbers (expt 2)}
\visible<3->{\item Diminishing returns for vagueness as gap size grows very large (expt 2)}
\end{itemize} }

\frame{\frametitle{Issues to address}
\begin{itemize}
\visible<1->{\item Potential for vagueness not realised? (expts 1 \& 2)}
\visible<2->{\begin{itemize}\item Definite articles uniquely identify}
\visible<3->{\item Two squares means no borderline case}
\visible<4->{\item Solution: use indefinite articles; use $>2$ squares\end{itemize}}
\visible<5->{\item Vagueness confounded with absence of a number in the instructions?}
\visible<6->{\begin{itemize} \item V: \{few, many\}; P: \{5, twenty\} }
\visible<7->{\item Solution: factorially manipulate instruction format 2 x 2}
\visible<8->{ \begin{table}[htdp]
%\caption{default}
\begin{center}
\begin{tabular}{c|c|c}
                     & Vague                              &  Precise \\ 
                     \hline
Numerical  &  \{about 20, about 30\}  &  \{16, 34\} \\ 
\hline
Verbal         & \{few, many\}                    &  \{fewest, most\}  \\
\hline
\end{tabular}
\end{center}
\label{default}
\end{table}%
}\end{itemize} \end{itemize} }
%; Precise-numerical: \{16, 34\}, Precise-verbal \{fewest, most\}, Vague-verbal: \{few, many\}; Vague-numerical \{about 20, about 40\}  }\end{itemize} \end{itemize} }
            

\section[Address problems]{Experiment three (addresses problems)}

\frame{
            \frametitle{Stimulus, experiment three}
            \begin{figure}
            \includegraphics[width=.7\textwidth]{/Users/mjgmac562/Documents/vagueness-folder/writing/figures/drawingexp3}
            \end{figure}
            }


\frame{
            \frametitle{Operationalising borderline cases}
            \begin{figure}
            \includegraphics[width=.7\textwidth]{/Users/mjgmac562/Documents/vagueness-folder/writing/figures/borderline}
            \end{figure}
            Number combinations: \{$6,15,24$\}, \{$16,25,34$\}, \{$26,35,44$\}, \{$36,45,54$\}
\begin{itemize} \item            Numerical-precise: Choose the square with 16 dots
           \item Numerical-vague: Choose a square with about 20 dots
           \item  Verbal-precise: Choose the square with fewest dots
            \item Verbal-vague: Choose a square with few dots\end{itemize}
            }


\frame{
            \frametitle{Experiment three (more squares \& use indefinites)}
            \begin{figure}
            \includegraphics[width=1\textwidth]{/Users/mjgmac562/Documents/vagueness-folder/e3/figures/borderlineresponseplottaskvgncolourpresentationfirstslide}
            \end{figure}
            }

\frame{
            \frametitle{Experiment three (more squares \& use indefinites)}
            \begin{figure}
            \includegraphics[width=1\textwidth]{/Users/mjgmac562/Documents/vagueness-folder/e3/figures/borderlineresponseplottaskvgncolourpresentationsecondslide}
            \end{figure}
            }
            
\frame{
            \frametitle{Experiment three (more squares \& use indefinites)}
            \begin{figure}
            \includegraphics[width=1\textwidth]{/Users/mjgmac562/Documents/vagueness-folder/e3/figures/borderlineresponseplottaskvgncolourpresentationthirdslide}
            \end{figure}
            }
           
           
\frame{
            \frametitle{Experiment three (more squares \& use indefinites)}
            \begin{figure}
            \includegraphics[width=1\textwidth]{/Users/mjgmac562/Documents/vagueness-folder/e3/figures/borderlineresponseplottaskvgncolourpresentationfourthslide}
            \end{figure}
            }
 
            
                    
\section[What does it all mean?]{Interpretation (what does it all mean?)}
\frame{
            \frametitle{Interpretation and Conclusions}
            \visible<1->{Summary of findings}
           \visible<2->{ \begin{itemize} 
            \item We did find some benefits for using some vague terms \ldots}
            \visible<3->{\item \dots but verbal format is largely responsible}
            \end{itemize}
            \visible<4->{Summary of explanations}
            \visible<5->{\begin{itemize} 
            \item Hierarchy of gist account}
           \visible<6->{ \item Heuristic / Systematic account
            \end{itemize} }
            \visible<7->{Lessons for NLG}
            \begin{itemize}
            \visible<8->{\item Data from human comprehenders can help NLG systems make better decisions}
            \visible<9->{\item Vagueness is really hard to pin down experimentally}
            \visible<10->{\item Using vague terms can make a big difference for cognitive load on comprehenders}
            \end{itemize} 
            }
%\frame{\titlepage}
%
%\section[Outline]{}
%\frame{\tableofcontents}
%
%\section{Introduction}
%\subsection{Overview of the Beamer Class}
%\frame
%{
%  \frametitle{Features of the Beamer Class}
%
%  \begin{itemize}
%  \item<1-> Normal LaTeX class.
%  \item<2-> Easy overlays.
%  \item<3-> No external programs needed.      
%  \end{itemize}
%}

%\renewcommand*{\refname}{}
\section[]{Take home message}
\frame{
\frametitle{Take home message}
\visible<2->{Vagueness does make things easier in some situations, but really it is non-numerical format that does the heavy lifting.}
\begin{center}
\visible<1->{\includegraphics[width=.4\textwidth]{/Users/mjgmac562/Documents/vagueness-folder/writing/figures/barometer}}
\includegraphics[width=.4\textwidth]{/Users/mjgmac562/Documents/vagueness-folder/writing/figures/digitalbarometer}\\
~~\\
\visible<3->{Thank you for listening. 5 mins for Q \& A}
\end{center}
}


\frame{
            \frametitle{Next}
            \begin{figure}
            \includegraphics[width=1\textwidth]{/Users/mjgmac562/Documents/vagueness-folder/writing/figures/lifesavinggraph}
            \end{figure}
            }



\renewcommand*{\refname}{}
\section*{}
\frame{
\frametitle{References}
\bibliographystyle{apacite}
\tiny{\bibliography{/Users/mjgmac562/Documents/vagueness-folder/writing/bibliography/vag.bib} }
}


\end{document}
