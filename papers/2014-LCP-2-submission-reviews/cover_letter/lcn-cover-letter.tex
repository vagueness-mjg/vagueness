%basic cover letter template
\documentclass{letter}
\usepackage{amssymb,amsmath}
\usepackage{graphicx}


\oddsidemargin=.2in
\evensidemargin=.2in
\textwidth=5.9in
\topmargin=-.5in
\textheight=9in

\address{Matthew Green and Kees van Deemter\\Department of Computing Sciences\\University of Aberdeen\\Aberdeen, AB24 3UE\\tel: 07896310566\\ email: mjgreen@abdn.ac.uk}

\newcommand {\qed}{\mbox{$\Box$}}
\renewcommand {\iff}{\Longleftrightarrow}
\newcommand {\R}{\mathbb{R}}
\newcommand {\N}{\mathbb{N}}
\newcommand {\Q}{\mathbb{Q}}
\newcommand {\Z}{\mathbb{Z}}

\newcommand {\sub}{\mbox{SB}}


\begin{document}
\begin{letter}{Editor\\
Language, Cognition and Neuroscience}


\opening{Dear Editor,}

We are pleased to submit our manuscript ``Vagueness as cost reduction" for consideration for publication in Language, Cognition and Neuroscience. The manuscript is co-authored by Matthew J. Green and Kees van Deemter (Department of Computing Science, University of Aberdeen). The manuscript is not currently under review with any other journal.

A previous version of the manuscript was submitted to what was then Language and Cognitive Processes on 01-Nov-2011, with Manuscript ID PLCP-2011-OP-9118. We received helpful reviews in response to that submission and we will refer to those reviews below. The original submission contained experiments 1 to 3 of the current submission, which adds experiments 4 and 5.

Vagueness is pervasive in everyday language use, but its pervasiveness  is not well understood. From some perspectives, including that of some game theorists, a precise, crisp term should always be preferable to a vague equivalent. Vague terms create potential misunderstandings, and can be considered to impose an efficiency loss. However, this efficiency loss is unlikely to have arisen unless vague terms confer advantages as well as disadvantages. At present there is little published work that investigates the utility of vagueness from an empirical perspective. The present manuscript seeks to make a contribution to empirical work on the utility of vagueness.

Reviews of the original submission all mentioned that the contrast we established between vague and crisp terms was also consistent with a difference between `comparison' and `matching' verification procedures (or selection algorithms). To spell this out, and explain how we addressed this concern, it will be necessary to describe our experimental setup briefly. Participants were presented with a set of dot arrays, and required to identify one set in response to an instruction of the form: ``Choose a square with $n$ dots" where $n$ was either a numerical quantifier (e.g., 5) or a verbal quantifier (e.g., few).  The concern of the reviewers was that an instruction like ``Choose a square with 5 dots", which we had cast as the crisp condition, invited a `matching' task where the participant had to match the quantity in the instruction to a quantity in the stimulus, whereas an instruction like ``Choose a square with few dots", which we had cast as the vague condition, invited a `comparison' task where the participant had to compare the dot arrays with each other to satisfy the instruction. In other words our crisp / vague dimension was confounded with the matching / comparison verification method.

One reviewer focused on another potential confound - that the difference which we had described as crisp / vague was also a difference between numerals in the crisp cases and verbal quantifiers in the vague cases.  This reviewer suggested that numerals invited counting where verbal quantifiers invited estimation, and that this difference could explain what we had described as effects of vagueness.  We have added material to the Introduction addressing these confounds, as suggested by reviewers.

Reviewers suggested that we should carry out further experiments to address these confounds. We did this in experiments 4 and 5 of the current submission.  Experiment 4 used numerals in each condition - we devised crisp and vague forms of the instructions, and comparison and matching conditions for each of the crisp and vague forms. In Experiment 5 we did the same for verbal quantifiers. In this way we crossed vagueness with verification method at each level of the numeral / verbal quantifier dimension.

The findings of the new experiments 4 and 5 are that neither the verification method nor the numerical / verbal format of the instruction can fully explain the effects of vagueness that we identified: at each level of the numerical / verbal format dimension vagueness exerted benefits in the comparison conditions but exerted disadvantages in the matching conditions.

In closing we would like to raise the possibility of inviting the original reviewers to review the current submission, in order to capitalise on the work already done in examining a previous version of the manuscript, but of course we understand that this is the editor's decision.

\closing{Sincerely,\\
\includegraphics[width=3cm]{mysignature.pdf}
}
Matthew Green\\
Kees van Deemter
\end{letter}

\end{document}

