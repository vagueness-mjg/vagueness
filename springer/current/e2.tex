\marginnote{D-Exp-2}

The main aim of Experiment 2 was to see whether vagueness would exert beneficial effects when all conditions used numerals in the instructions, and when there were vague and crisp versions of the instructions for both comparison and matching strategies. The main changes from Experiment 1 were that the human selection task was explicitly controlled (i.e., in whether it amounted to matching or comparison), and that all conditions were constrained to mention a number. We used the same arrays as in Experiment 1 (an example stimulus is given in Figure \ref{Experiment1and2examplestimulus}). 38 participants were recruited. We used a 2 x 2 factorial manipulation of vagueness and selection task (see Table \ref{Instructions for e2}). On each trial an instruction was presented: participants pressed a key to dismiss the instruction, at which time the dot arrays were presented until the participant responded, and the response time and choice were recorded. Table \ref{Instructions for e2} shows the instructions for each condition. Note the difference between ``fewer than 20'' and ``far fewer than 20'': whereas the former cannot have borderline cases (i.e., for each number it is clear whether the number is smaller than 20 or not), the latter can.

\begin{table}
\centering
\caption{Experiment 2: Instructions arranged by condition. The instructions given in the table started with ``Choose a square with \ldots"} 
\label{Instructions for e2}
\begin{tabular}{cccll}
\hline\noalign{\smallskip}
Item & Quantity & Selection & Crisp & Vague \\ 
\noalign{\smallskip}\hline\noalign{\smallskip}
06:15:24 & Small & Comparison & fewer than 20 dots & far fewer than 20 dots \\ 
06:15:24 & Small & Matching & 6 dots & about 10 dots \\ 
06:15:24 & Large & Comparison & more than 10 dots & far more than 10 dots \\ 
06:15:24 & Large & Matching & 24 dots & about 20 dots \\ 
\noalign{\smallskip}\hline\noalign{\smallskip}
16:25:34 & Small & Comparison & fewer than 30 dots & far fewer than 30 dots \\ 
16:25:34 & Small & Matching & 16 dots & about 20 dots \\ 
16:25:34 & Large & Comparison & more than 20 dots & far more than 20 dots \\ 
16:25:34 & Large & Matching & 34 dots & about 30 dots \\ 
\noalign{\smallskip}\hline\noalign{\smallskip}
26:35:44 & Small & Comparison & fewer than 40 dots & far fewer than 40 dots \\ 
26:35:44 & Small & Matching & 26 dots & about 30 dots \\ 
26:35:44 & Large & Comparison & more than 30 dots & far more than 30 dots \\ 
26:35:44 & Large & Matching & 44 dots & about 40 dots \\ 
\noalign{\smallskip}\hline\noalign{\smallskip}
36:45:54 & Small & Comparison & fewer than 50 dots & far fewer than 50 dots \\ 
36:45:54 & Small & Matching & 36 dots & about 40 dots \\ 
36:45:54 & Large & Comparison & more than 40 dots & far more than 40 dots \\ 
36:45:54 & Large & Matching & 54 dots & about 50 dots \\ 
\noalign{\smallskip}\hline
\end{tabular}
\end{table}

\subsection{Hypotheses (Experiment 2)}

For Experiment 2, we formulated the following hypotheses:

\begin{description}
\item [Hypothesis 1] Vague instructions should result in faster responses than crisp instructions (main effect of vagueness).
\item [Hypothesis 2] Instructions that allow comparison should result in faster responses than instructions that necessitate matching (main effect of selection task).
\item [Hypothesis 3] Vagueness effect should differ between comparison and matching: The effects of vagueness should differ when the selection task is comparison versus when it is matching (interaction effect selection x vagueness).
\end{description}

\subsection{Results (Experiment 2)}

%We used the method proposed by \citet{Levy:MainEffectsInteractions} to test for the main effect of Vagueness in the presence of its higher-order interactions.

Response times were trimmed at 2.5 SD separately for each subject, leading to the loss of 204 trials (2.8\% of the trials). Condition means for the remaining (logged) RTs are plotted in Figure \ref{resultsD-exp-2}. A linear mixed model was constructed for the logged response times, with sum-coded vagueness, instruction format, (and their interaction), and item as fixed effects, and the same effects as slopes over participant for random effects.

\begin{figure}[htbp]
\centering
\includegraphics[width=\textwidth]{figures/De2-rtplot-1.pdf}
\caption{Mean response times by condition for Experiment 2 where all instructions were numeric}
\label{resultsD-exp-2}
\end{figure}

\begin{description} %
	\item [Test of Hypothesis 1] Vague instructions resulted in faster responses than crisp instructions on average. However this difference was not significant in the full model ($\beta=-0.0057$, $se=0.0137$, $t=-0.42$, $p=0.678$). Using Levy's method \citep{Levy:MainEffectsInteractions} to test for main effects in the presence of higher-order interactions, by doing model comparison between a null model that included all interaction terms involving Vagueness but leaving out a term for the main effect of Vagueness, against a full model that differed only by including Vagueness as a main effect, showed that the full model was no better than the reduced model (df=1, p=0.6764) - consituting more evidence that Vagueness did not exert a significant main effect on response times. 
	\item [Test of Hypothesis 2] comparison instructions resulted in faster responses than matching instructions, and the difference was significant ($\beta=0.1618$, $se=0.0255$, $t=6.34$, $p<0.001$).
	\item [Test of Hypothesis 3] Although Vagueness did not exert a significant main effect, Vagueness did exert effects in interactions with some other variables: the interaction between Vagueness and Selection task  was significant ($\beta=0.1306$, $se=0.0205$, $t=6.38$, $p<0.001$), suggesting that Vagueness speeded RTs in the comparison condition but slowed them down in the matching task. 
	Separate analyses at each level of selection \ldots
	%provided evidence that vague instructions resulted in faster responses than crisp instructions for the comparison condition (p<0.01); and slower responses than crisp instructions in the matching condition (p<0.01). 
%
%Separate analyses were conducted testing for effects of vagueness at each level of the selection task.
%Within the comparison task vagueness significantly speeded response times compared with crisp controls ($\beta=-0.07$, $se=0.02$, $t=-3.5$, $p<0.0012$). 
%Within the matching task vagueness significantly \emph{slowed} response times compared with crisp controls ($\beta=0.06$, $se=0.02$, $t=2.9$, $p<0.0061$). 
\end{description}

\subsection{Discussion (Experiment 2)}
The cost reduction account predicted that there should be a significant main effect of Vagueness such that responses would be faster for Vague intructions than for crisp instructions. We found that although there was a very small effect in that direction, the effect was not statistically significant. Models that differed only in the presence of Vagueness as a main effect were shown not to differ signigicantly in their explanatory value. However we did find that Vagueness exerted effects on other variables: Vgueness speeded RTs in the comparison task and slowed RTs in the matching task.


%In other words, the cost reduction account was wrong to predict significant main effect advantages for vagueness (although there was a non-significant trend in the \emph{direction} predicted by the cost reduction account), and wrong to predict that vagueness should be beneficial at each level of the selection task: however vagueness was significantly advantageous in the comparison task.
